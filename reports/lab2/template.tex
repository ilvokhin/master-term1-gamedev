\documentclass[12pt]{article}

\usepackage{fullpage}
\usepackage{multicol, multirow}
\usepackage{tabularx}
\usepackage{standalone}
\usepackage{listings}
\usepackage{ulem}
\usepackage{amsmath}
\usepackage{pdfpages}
\usepackage[utf8]{inputenc}
\usepackage[russian]{babel}

\newcommand{\StudentName}{Ильвохин Дмитрий}
\newcommand{\Group}{1O-106М}
\newcommand{\CourseName}{Программирование игр}
\newcommand{\LabNum}{1}
\newcommand{\Subject}{Арканоид}
\newcommand{\PrepName}{Аносова Н.\,П.}

\begin{document}

%\include{title} % title page

\lstset
{
        language=Python,
        basicstyle=\footnotesize,% basic font setting
        extendedchars=\true
}

\begin{flushright}
\Large{
	\CourseName \\
	Лабораторная работа №\,\LabNum \\
	<<\Subject>> \\
	%\StudentName, \Group \\
}
\end{flushright}

\subsection*{Задание}
Реализовать игру арканоид.

При столкновеинии ракетки и мяча угол отскока должен меняться в зависимости от места,
куда пришелся удар (удар по центу к изменению не приводит, чем ближе к краю,
тем более пологий угол отскока).

При ударе о ракетку во время достаточно быстрого ее движение угол отражения тоже должен меняться
(вектор движения мяча при отражении складывается с вектором движения ракетки, 
умноженным на некоторый коэффициент) для создания эффекта подкручивания мяча.

\subsection*{Практическая часть}

\subsection*{Выводы}
\end{document}

